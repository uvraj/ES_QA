\subsection{Bemessungskriterien}
\subsubsection{Anordnung von Leitungsschutzorganen}

Lösen Sie folgende Aufgabenstellungen:
\begin{enumerate}
    \item   \question{An welcher Stelle einer Leitung sind Überstromschutzorgane anzubringen?} \\\\
            An allen Stellen, an denen eine Änderung des 
            \begin{itemize}
                \item Querschnitts
                \item Verlegungsart
                \item Leitungsaufbaus
            \end{itemize}

            den zulässigen Dauerstrom $I_Z$ reduziert.
    \item   \question{Darf der N-Leiter mit einem eigenen Überstromschutzorgan abgesichert werden? \\ a) Wird empfohlen \\ b) Ja \\ c) Nein \\[\baselineskip] (Antwort + Begründung + Bedingungen)} \\\\
            Nein! Löst der LSS aus, so würde sich trotz eines ausgelösten LSS eine Spannung in Geräten befinden. Dies kann eine Gefahr bei der Fehlersuche darstellen.
    \item   \question{Darf der N-Leiter gemeinsam mit dem Überstromschutzorgan der Aussenleiter abgesschaltet werden? \\ a) Wird empfohlen \\ b) Ja \\ c) Nein \\[\baselineskip] (Antwort + Begründung + Bedingungen)} \\\\
            TODO
    \item   \question{Darf der PE-Leiter mit einem eigenen Überstromschutzorgan abgesichert werden? \\ a) Wird empfohlen \\ b) Ja \\ c) Nein \\[\baselineskip] (Antwort + Begründung + Bedingungen)} \\\\
            Nein, behinderung der Schutzmaßnahmen mit Schutzleiter.
    \item   \question{Darf der PE-Leiter gemeinsam mit dem Überstromschutzorgan der Aussenleiter abgesschaltet werden? \\ a) Wird empfohlen \\ b) Ja \\ c) Nein \\[\baselineskip] (Antwort + Begründung + Bedingungen)} \\\\
            Nein, behinderung der Schutzmaßnahmen mit Schutzleiter.
    \item   \question{Darf der Überstromschutz vom Kurzschlussschutz getrennt werden? (+Beispiel) An welcher Stelle der Leitung können die jeweiligen Schutzorgane angebracht werden?} \\\\
            Ja, beispielsweise bei Motoren. Thermorelais + Schmelzsicherungen. Der Kurzschlusschutz muss grundsätzlich an dem Leitungsanfang platziert werden.
            (Er lässt sich unter bestimmten Bedingungen verschieben) Der Überstromschutz kann beliebig verschoben werden.
\end{enumerate}

\subsubsection{Schaltanlage / graphische Symbole}
Lösen Sie die unten aufgelisteten Aufgabenstellungen.

\begin{enumerate}
    \item   \question{Zeichnen Sie das Symbol einer Schmelzsicherung für Drehstromanschluss (einpolige und mehrpolige Darstellung)}
    \item   \question{Zeichnen Sie das Symbol eines Leitungsschutzschalters für Drehstromanschluss (einpolige und mehrpolige Darstellung)}
\end{enumerate}