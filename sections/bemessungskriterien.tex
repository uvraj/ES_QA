\subsection{Bemessungskriterien}
\subsubsection{Anordnung von Leitungsschutzorganen}

Lösen Sie folgende Aufgabenstellungen:
\begin{enumerate}
    \item   \question{An welcher Stelle einer Leitung sind Überstromschutzorgane anzubringen?}
    \item   \question{Darf der N-Leiter mit einem eigenen Überstromschutzorgan abgesichert werden? \\ a) Wird empfohlen \\ b) Ja \\ c) Nein \\[\baselineskip] (Antwort + Begründung + Bedingungen)}
    \item   \question{Darf der N-Leiter gemeinsam mit dem Überstromschutzorgan der Aussenleiter abgesschaltet werden? \\ a) Wird empfohlen \\ b) Ja \\ c) Nein \\[\baselineskip] (Antwort + Begründung + Bedingungen)}
    \item   \question{Darf der PE-Leiter mit einem eigenen Überstromschutzorgan abgesichert werden? \\ a) Wird empfohlen \\ b) Ja \\ c) Nein \\[\baselineskip] (Antwort + Begründung + Bedingungen)}
    \item   \question{Darf der PE-Leiter gemeinsam mit dem Überstromschutzorgan der Aussenleiter abgesschaltet werden? \\ a) Wird empfohlen \\ b) Ja \\ c) Nein \\[\baselineskip] (Antwort + Begründung + Bedingungen)}
    \item   \question{Darf der Überstromschutz vom Überlastschutz getrennt werden? (+Beispiel) An welcher Stelle der Leitung können die jeweiligen Schutzorgane angebracht werden?}
\end{enumerate}

\subsubsection{Schaltanlage / graphische Symbole}
Lösen Sie die unten aufgelisteten Aufgabenstellungen.

\begin{enumerate}
    \item   \question{Zeichnen Sie das Symbol einer Schmelzsicherung für Drehstromanschluss (einpolige und mehrpolige Darstellung)}
    \item   \question{Zeichnen Sie das Symbol eines Leitungsschutzschalters für Drehstromanschluss (einpolige und mehrpolige Darstellung)}
\end{enumerate}