\subsection{Licht und Wahrnehmung}
Beantworten Sie folgende Fragen zum Thema Licht und Wahrnehmung.
\begin{enumerate}
    \item   \question{In welchem Wellenlängenbereich der elektromagnetischen Strahlung kann das menschliche Auge Licht wahrnehmen?}
    \item   \question{Bei welcher Wellenlänge liegt die größte Hellempfindlichkeit für\\ a) Tagsehen \\b) Nachtsehen}
    \item   \question{Was versteht man unter Akkomodation des Auges?}
    \item   \question{Was versteht man unter Adaption des Auges und welche Abläufe im Auge ermöglichen die Adaption?}
    \item   \question{Wie entstehen die Farben aus den Grundfarben über additive Farbmischung?}
    \item   \question{Wie entstehen die Farben aus den Grundfarben über subtraktive Farbmischung?}
    \item   \question{Was versteht man in der Lichttechnik unter einem kontinuierlichen Spektrum. Nennen Sie Beispiele für Lichtquellen mit kontinuierlichem Spektrum.}
    \item   \question{Was versteht man in der Lichttechnik unter einem diskreten Spektrum? Nennen Sie Beispiele für Lichtquellen mit diskretem Spektrum.}
    \item   \question{Was versteht man unter Farbtemperatur: Wie ist sie definiert und in welcher Einheit wird sie angegeben?}
    \item   \question{Was versteht man unter dem Farbwiedergabeindes $R_a$? Wie wird der Farbwiedergabeindex berechnet?}
    \item   \question{Was bedeutet eine Kennzeichnung mit den drei Ziffern 950 auf dem Sockel eines Leuchtmittels?}
\end{enumerate}

\subsubsection{Lichttechnische Größen}
Arbeiten Sie folgende Aufgabenstellung genau und zielführend durch:

\begin{enumerate}
    \item   \question{Nennen sie die vier lichttechnischen Grundgrößen und ihre Einheiten sowie die Bedeutung dieser Größen.}
    \item   \question{Wie heißt die lichttechnische Größe und Einheit, mit der die gesamte, von einer Lichtquelle abgegebene und vom Auge bewertete Strahlungsleistung gemessen wird?}
    \item   \question{Wie errechnet sich die Lichtausbeute einer Lichtquelle (Formel angeben)?}
    \item   \question{Auf welche lichttechnische Größe bezieht sich die Energieeffizienzklasse, die entsprechend der EU-Richtlinie auf der Verpackung von Leuchtmitteln angegeben ist?}
    \item   \question{Wie heißt die lichttechnische Größe und Einheit, mit der die gesamte, auf einer Fläche auftreffende und vom Auge bewertete Strahlungsleistung $\phi$ im Verhältnis zur Flächengröße gemessen $A$ wird?}
    \item   \question{Wie errechnet sich die Beleuchtungsstärke aus $\phi$ (Formel angeben)?}
    \item   \question{Wie errechnet sich die horizontale Beleuchtungsstärke aus $I$?}
    \item   \question{Wie errechnet sich die vertikale Beleuchtungsstärke aus $I$?}
    \item   \question{Was ist eine Lichtverteilungskurve? Erklären Sie anhand der C-Ebene den Zusammenhang zwischen Lichtverteilungskurve und Lichtverteilungskörper.}
    \item   \question{Erklären Sie den Zusammenhang zwischen Lichtstrom und Lichtstärke.}
    \item   \question{Wie groß ist die Beleuchtungsstärke, wenn ein Lichtstrom $\phi = \SI{800}{lm}$ gleichmäßig und normal auf eine Fläche von $A = \SI{6}{m^2}$ auftrifft?}
\end{enumerate}

