\subsection{Licht und Wahrnehmung}
Beantworten Sie folgende Fragen zum Thema Licht und Wahrnehmung.
\begin{enumerate}
    \item   \question{In welchem Wellenlängenbereich der elektromagnetischen Strahlung kann das menschliche Auge Licht wahrnehmen?}\\\\
            380-780nm
    \item   \question{Bei welcher Wellenlänge liegt die größte Hellempfindlichkeit für\\ a) Tagsehen \\b) Nachtsehen} \\\\
            550 und 505 nm respektive
    \item   \question{Was versteht man unter Akkomodation des Auges?}\\\\
            Anpassung an Tiefenverhältnisse. Durch eine "flexible" Linse im Auge lässt sich die Brennweite verändern, hierdurch eine Änderung 
            der Fokussierung möglich.
    \item   \question{Was versteht man unter Adaption des Auges und welche Abläufe im Auge ermöglichen die Adaption?} \\\\
            Anpassung an Lichtverhältnisse. Die Iris wirkt wie eine Blende bei einer Kamera. Bei hellen Szenen schließt sie sich, bei dunklen öffnet sie sich.
    \item   \question{Wie entstehen die Farben aus den Grundfarben über additive Farbmischung?}\\\\
            Ausgehend von einem reinen schwarz wird die Ergebnisfarbe heller, je mehr Farbanteile hinzugefügt werden.
    \item   \question{Wie entstehen die Farben aus den Grundfarben über subtraktive Farbmischung?}\\\\
            Ausgehend von einem reinen Weiß (alle Frequenzanteile vorhanden) wird die Ergebnisfarbe umso dünkler, je mehr Farbanteile herausgefilter (subtrahiert) werde. 
    \item   \question{Was versteht man in der Lichttechnik unter einem kontinuierlichen Spektrum. Nennen Sie Beispiele für Lichtquellen mit kontinuierlichem Spektrum.} \\\\
            Ein kontinuierliches Spektrum beschreibt eine durchgängige/konsistente Spektralverteilung. Sie weist keine scharfen Kanten/Spektrallinien auf.
    \item   \question{Was versteht man in der Lichttechnik unter einem diskreten Spektrum? Nennen Sie Beispiele für Lichtquellen mit diskretem Spektrum.} \\\\
            Ein diskretes Spektrum beschreibt eine unterschiedliche Spektralverteilung. Es sind Diskontinuitäten (scharfe Spektrallinien) zu sehen.
    \item   \question{Was versteht man unter Farbtemperatur: Wie ist sie definiert und in welcher Einheit wird sie angegeben?}\\\\
            Die Farbtemperatur beschreibt den Farbeindruck einer Lichtquelle. Sie wird in Kelvin angegeben. Sie ist gleich der Temperatur, die man 
            auf einen schwarzen Körper anbringen muss, um den gleichen Farbeindruck zu erhalten.
    \item   \question{Was versteht man unter dem Farbwiedergabeindes $R_a$? Wie wird der Farbwiedergabeindex berechnet?}\\\\
            Der Farbwidergabeindex gibt an, wie Farbecht Farben unter künstlichem Licht dargestellt werden. Der Farbwiedergabeindex 
            wird berechnet, indem die Farbverschiebung von 8 genormten Farben unter der zu untersuchenden Lichtquelle und einer Referenzlichtquelle 
            berechnet wird. Die Verschiebung wird mit einem Gewicht 1/8 pro Farbe aufsummiert.
    \item   \question{Was bedeutet eine Kennzeichnung mit den drei Ziffern 950 auf dem Sockel eines Leuchtmittels?}\\\\
            Farbwiedergabe 90\%, Stufe 1A. Farbtemperatur 5000K.
\end{enumerate}

\subsubsection{Lichttechnische Größen}
Arbeiten Sie folgende Aufgabenstellung genau und zielführend durch:

\begin{enumerate}
    \item   \question{Nennen sie die vier lichttechnischen Grundgrößen und ihre Einheiten sowie die Bedeutung dieser Größen.}\\\\
            Lichtstrom $\phi, lm$, Lichtstärke $I, cd$, Beleuchtungsstärke $E, lux$ und Leuchtdichte $L, \frac{cd}{m^2}$.
    \item   \question{Wie heißt die lichttechnische Größe und Einheit, mit der die gesamte, von einer Lichtquelle abgegebene und vom Auge bewertete Strahlungsleistung gemessen wird?}\\\\
            Lichtstrom $\phi, lm$.
    \item   \question{Wie errechnet sich die Lichtausbeute einer Lichtquelle (Formel angeben)?}\\\\
            $$\eta = \frac{\phi}{P}$$
    \item   \question{Auf welche lichttechnische Größe bezieht sich die Energieeffizienzklasse, die entsprechend der EU-Richtlinie auf der Verpackung von Leuchtmitteln angegeben ist?}\\\\
            Auf die Lichtausbeute.
    \item   \question{Wie heißt die lichttechnische Größe und Einheit, mit der die gesamte, auf einer Fläche auftreffende und vom Auge bewertete Strahlungsleistung $\phi$ im Verhältnis zur Flächengröße gemessen $A$ wird?}\\\\
            $$E, lux$$
    \item   \question{Wie errechnet sich die Beleuchtungsstärke aus $\phi$ (Formel angeben)?}\\\\
            $$E = \frac{\phi}{A}$$
    \item   \question{Wie errechnet sich die horizontale Beleuchtungsstärke aus $I$?}\\\\
            $$E_H = \frac{I_V}{h^2}\cdot \cos\left(\alpha\right)^3$$
    \item   \question{Wie errechnet sich die vertikale Beleuchtungsstärke aus $I$?}\\\\
            $$E_V = \frac{I_V}{r^2}\cdot \sin\left(\alpha\right)$$
    \item   \question{Was ist eine Lichtverteilungskurve? Erklären Sie anhand der C-Ebene den Zusammenhang zwischen Lichtverteilungskurve und Lichtverteilungskörper.}\\\\
            Mit der LVK kann die Lichtstärke für jeden beliebigen Raumwinkel dargestellt werden?
    \item   \question{Erklären Sie den Zusammenhang zwischen Lichtstrom und Lichtstärke.} \\\\
            $$I = \frac{d\Omega}{d\phi}$$
    \item   \question{Wie groß ist die Beleuchtungsstärke, wenn ein Lichtstrom $\phi = \SI{800}{lm}$ gleichmäßig und normal auf eine Fläche von $A = \SI{6}{m^2}$ auftrifft?}
\end{enumerate}

