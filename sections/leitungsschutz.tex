\subsection{Leitungsschutz}
Beantworten Sie folgende Fragen:

\begin{enumerate}
    \item   \question{Wodurch kann in Verteilungsnetzen (EVU-Netz, Verbraucheranlage) Überstrom auftreten?} \\\\
            Durch den Anschluss einer Überlast oder eines Kurzschlusses.
    \item   \question{Wie erfolgt der Schutz gegen Überstrom in Verbraucheranlagen?}\\\\
            Durch den Einbau von Überstrom-Schutzorganen.
    \item   \question{Welche zwei Arten von Überstromschutzorganen kennen Sie?}\\\\
            Schmelzsicherungen sowie Leitungssschutzschalter.
\end{enumerate}

\subsubsection{Schmelzsicherungen}
Beantworten Sie die unten aufgelisteten Aufgabestellungen. Eine Begründung Ihrer Entscheidung ist essenziell.

\begin{enumerate}
    \item   \question{Welche Bauarten werden bei Schmelzsicherungen unterschieden?} \\\\
            Stöpselsicherungen und NH-Sicherungen.
    \item   \question{Nennen Sie die wichtigsten Kenngrößen von Schmelzsicherungen} \\\\
            Bemessungsspannung, Bemessungsstrom, Bemessungsschaltvermögen und Betriebsklasse.
            Bemssungspannung: Die Spannung, für die alle Teile der Sicherung ausgelegt sind. Bemessungsstrom: Der Strom, den die Sicherung ohne Schmelzen dauerhaft
            führen muss. Das Bemessungsschaltvermögen bezeichnet jenen Kurzschlusstrom, den die Sicherung ohne Gefahr abschalten kann. Die Betriebsklasse gibt Auskunft über die Verwendbarkeit und 
            ZEit-Strom-Kennlinie.
            \LARGE BEMESSUNGSSCHALTVERMÖGEN STATT AUSSCHALTVERMÖGEN \normalsize
    \item   \question{Über welche Prüfströme wird die Auslösecharakteristik und Fertigungstoleranz einer Schmelzsicherung festgelegt? Wie sind diese Prüfströme definiert?} \\\\
            Durch den kleinen und großen Prüfstrom. Der kleine Prüfstrom ist jener Strom, bei dem die Schmelzsicherung innerhalb der konventionellen Prüfdauer nicht abschalten darf.
            Der große Prüfstrom bezeichnet jenen Strom, bei dem die Schmelzsicherung innerhalb der konventionellen Prüfdauer abschalten muss.
    \item   \question{Erklären Sie Aufbau und Funktion einer Stöpselsicherung (+Skizze)} \\\\
            Der Sicherungseinsatz selbst besteht aus 2 Kontakten, dem Porzellankörper, dem Schmelzleiter, dem Anzeigedraht, dem Anzeigeplättchen, der Anzeigefeder und aus dem Quarzsand, 
            der zur Lichtbogenlöschung dient.

            Das System (DIAZED/NEOZED), in dem der Einsatz verwendet wird, besteht aus der Schraubkappe, dem Einsatz selbst, dem Paßeinsatz, dem Isolierring und aus dem Sockel.

    \item   \question{Erklären Sie Aufbau und Funktion einer NH-Sicherung (+Skizze)}\\\\
            Die Schmelzsicherung selbst besteht aus dem Sicherungskörper selbst, aus zwei Kontakten, aus einen Anzeigedraht, aus einer Anzeigefeder, aus dem Schmelzleiter, aus dem Anzeigeplättchen und aus dem 
            Anzeigedraht.

    \item   \question{Worüber gibt die Betriebsklasse einer Schmelzsicherung auskunft?} \\\\
            Sie gibt über die Verwendbarkeit und über ihre Zeit-Strom-Kennlinie Auskunft.
    \item   \question{Nennen Sie 3 Vorteile eines NH-Trenners gegenüber Stöpselsicherungen.} \\\\
            Massivere Kontakte, was zu einer kleineren Erwärmung führt. NH-Sicherungen können auch besonders hohe Kurzschlusströme abschalten.
            Sie sind für den elektrotechnischen Laien normalerweise nicht zugänglich, was zu einer höheren Sicherheit führen könnte.

    \item   \question{Unter welchen Bedingungen werden Schmelzsicherungen gegenüber Leitungsschutzschaltern bevorzugt? (+Beispiele)} \\\\
            Wenn die Selektivität im Vordergrund steht. 
    \item   \question{Erklären Sie Aufbau und Funktion eines Leitungsschutzschalters. Wie erfolgt die Auslösung bei einem Leitungsschutzschalter.} \\\\
            Der Leitungsschutzschalter besitzt eine thermische und eine magnetische Auslösung. Die magnetische Auslösung wird durch eine Elektromagnetspule realisiert. Die thermische 
            durch einen Bimetallauslöser. Fließt mehr als der Nichtauslösestrom, so erwärmt sich das Bimetall, was zu einer Biegung führt. Der LSS löst aus. Bei einem Kurzschluss zieht die Spule an, durch einen 
            Mechanismus wird der Stromkreis unterbrochen.
    \item   \question{Nennen Sie Vorteile von Leitungsschutzschaltern gegenüber Schmelzsicherungen.}\\\\
            Leitungsschutzschalter können nach Behebung des Fehlers wieder eingeschaltet werden. Sie benötigen keinen Austausch, was zu weniger 
            Abfall führt. Außerdem können Sie bei Kurzschlüsse schneller abschalten. Sie sind außerdem nicht einfach überbrückbar/flickbar, was einen Vorteil angesichts der Sicherheit 
            darstellen könnte.
    \item   \question{Über welche Prüfströme  wird die Auslösecharakteristik und Fertigungstoleranz eines Leitungsschutzschalters festgelegt. Wie sind diese Prüfströme definiert?} \\\\
            Über den Nichtauslösestrom, dem Auslösestrom und den Sofortauslösestrom. Der Nichtauslösestrom ist jener Strom, bei dem der LSS nicht abschaltet. 1.13IN. Der Auslösestrom ist jener Strom,
            bei dem der LSS innerhalb einer festgelegten Prüfdauer auslöst.
    \item   \question{Wie unterscheiden sich Leitungsschutzschalter der Typen B, C und D in ihrer Auslösecharaktieristik (+Skizze)?} \\\\
            Sie unterscheiden sich nur durch die magnetischen Auslöser. Der thermische Auslöser ist bei allen gleich.
    \item   \question{Ein Winkelschleifer hat einen Nennstrom von 6A und nimmt beim Einschalten kurzzeitig den 8fachen Strom auf. Wird ein klagloser Betrieb möglich sein, wenn der Stromkreis mit einem 13A LS-Schalter vom Typ B abgesichert ist oder schlagen Sie einen anderen Leitungsschutzschalter vor? (+Begründung)} \\\\
            Für Handgeräte wird ein Winkelschleifer nach Type C empfohlen.
    \item   \question{Worauf muss man bei der Verwendung von LS-Schaltern der Type D im genullten Netz achten?} \\\\
            Es muss darauf geachtet werden, dass die Ausschaltbedingung trotz erhöhtem Ausschaltstrom erfüllt bleibt.

    \item   \question{Was versteht man unter der Energiebegrenzungsklasse eines Leitungsschutzschalters? Welche Auswirkung hat eine hohe Energiebegrenzungsklasse?} \\\\
            Die Energiebegrenzungsklasse ist ein Maß für die Durchlassenergie, die ein LSS benötigt, um den Stromkreis zu öffnen. Desto höher, desto schneller Schaltet der LSS 
            (bei Kurzschluss) aus.
\end{enumerate}

\subsubsection{Ausschaltstrom / Selektivität}
Lösen Sie folgende Aufgabenstellungen zum Thema Ausschaltstrom / Selektivität

\begin{enumerate}
    \item   \question{Wie ist der Ausschaltstrom von Überstromschutzorganen definiert?} \\\\
            Der Ausschaltstrom ist jener Strom, bei dem das ÜSO so schnell auslösen muss, dass keine Gefahr durch direktes Berühren entsteht.
    \item   \question{Was versteht man unter Selektivität von Überstromschutzorgangen? Unter welcher Voraussetzung ist die Selektivität gegeben? Nennen Sie 2 einfache Regeln zur Selektivität.} \\\\
            Selektivität bedeutet, dass beim Fehlerfall nur jenes ÜSO auslöst, welches sich unmittelbar in der Nähe des Fehlers befindet. Dies erhöht die Betriebssicherheit und Verfügbarkeit einer 
            elektrischen Anlage. Es gelten folgende Regeln
    
    \begin{itemize}
        \item Schmelzsicherungen sind bei einem Bemessungstromverhältnis von 1/1,6 selektiv 
        \item Leitungsschutzschalter sind nur im Überstrombereich bei einem Bemessungsstromverhältnis von 1/1,6 selektiv
        \item Bei einem Kurzschluss sind LSS nicht selektiv.
        \item LSS der Energiebegrenzungsklasse 3 bis 20A sind mit Schmelzsicherungen ab 50A sicher selektiv, wenn der max. Kurzschlussstrom nicht größer als 1kA ist.
    \end{itemize}
\end{enumerate}