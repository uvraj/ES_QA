\subsection{Leitungsschutz}
Beantworten Sie folgende Fragen:

\begin{enumerate}
    \item   \question{Wodurch kann in Verteilungsnetzen (EVU-Netz, Verbraucheranlage) Überstrom auftreten?}
    \item   \question{Wie erfolgt der Schutz gegen Überstrom in Verbraucheranlagen?}
    \item   \question{Welche zwei Arten von Überstromschutzorganen kennen Sie?}
\end{enumerate}

\subsubsection{Schmelzsicherungen}
Beantworten Sie die unten aufgelisteten Aufgabestellungen. Eine Begründung Ihrer Entscheidung ist essenziell.

\begin{enumerate}
    \item   \question{Welche Bauarten werden bei Schmelzsicherungen unterschieden?}
    \item   \question{Nennen Sie die wichtigsten Kenngrößen von Schmelzsicherungen}
    \item   \question{Über welche Prüfströme wird die Auslösecharakteristik und Fertigungstoleranz einer Schmelzsicherung festgelegt? Wie sind diese Prüfströme definiert?}
    \item   \question{Erklären Sie Aufbau und Funktion einer Stöpselsicherung (+Skizze)}
    \item   \question{Erklären Sie Aufbau und Funktion einer NH-Sicherung (+Skizze)}
    \item   \question{Worüber gibt die Betriebsklasse einer Schmelzsicherung auskunft?}
    \item   \question{Nennen Sie 3 Vorteile eines NH-Trenners gegenüber Stöpselsicherungen.}
    \item   \question{Unter welchen Bedingungen werden Schmelzsicherungen gegenüber Leitungsschutzschaltern bevorzugt? (+Beispiele)}
    \item   \question{Erklären Sie Aufbau und Funktion eines Leitungsschutzschalters. Wie erfolgt die Auslösung bei einem Leitungsschutzschalter.}
    \item   \question{Nennen Sie Vorteile von Leitungsschutzschaltern gegenüber Schmelzsicherungen.}
    \item   \question{Über welche Prüfströme  wird die Auslösecharakteristik und Fertigungstoleranz eines Leitungsschutzschalters festgelegt. Wie sind diese Prüfströme definiert?}
    \item   \question{Wie unterscheiden sich Leitungsschutzschalter der Typen B, C und D in ihrer Auslösecharaktieristik (+Skizze)?}
    \item   \question{Ein Winkelschleifer hat einen Nennstrom von 6A und nimmt beim Einschalten kurzzeitig den 8fachen Strom auf. Wird ein klagloser Betrieb möglich sein, wenn der Stromkreis mit einem 13A LS-Schalter vom Typ B abgesichert ist oder schlagen Sie einen anderen Leitungsschutzschalter vor? (+Begründung)}
    \item   \question{Worauf muss man bei der Verwendung von LS-Schaltern der Type D im genullten Netz achten?}
    \item   \question{Was versteht man unter der Energiebegrenzungsklasse eines Leitungsschutzschalters? Welche Auswirkung hat eine hohe Energiebegrenzungsklasse?}
\end{enumerate}

\subsubsection{Ausschaltstrom / Selektivität}
Lösen Sie folgende Aufgabenstellungen zum Thema Ausschaltstrom / Selektivität

\begin{enumerate}
    \item   \question{Wie ist der Ausschaltstrom von Überstromschutzorganen definiert?}
    \item   \question{Was versteht man unter Selektivität von Überstromschutzorgangen? Unter welcher Voraussetzung ist die Selektivität gegeben? Nennen Sie 2 einfache Regeln zur Selektivität.}
\end{enumerate}