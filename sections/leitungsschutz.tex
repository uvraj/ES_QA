\subsection{Leitungsschutz}
Beantworten Sie folgende Fragen:

\begin{enumerate}
    \item   \question{Wodurch kann in Verteilungsnetzen (EVU-Netz, Verbraucheranlage) Überstrom auftreten?} \\\\
            Durch den Anschluss einer niederohmigen Last oder durch einen Kurzschluss.
    \item   \question{Wie erfolgt der Schutz gegen Überstrom in Verbraucheranlagen?} \\\\
            Durch Überstrom-Schutzorgane.
    \item   \question{Welche zwei Arten von Überstromschutzorganen kennen Sie?} \\\\
            Schmelzsicherungen und Leitungsschutzschalter.
\end{enumerate}

\subsubsection{Schmelzsicherungen}
Beantworten Sie die unten aufgelisteten Aufgabestellungen. Eine Begründung Ihrer Entscheidung ist essenziell.

\begin{enumerate}
    \item   \question{Welche Bauarten werden bei Schmelzsicherungen unterschieden?} \\\\
            Es wird zwischen Stöpselsicherungen und NH-Sicherungen unterschieden.

    \item   \question{Nennen Sie die wichtigsten Kenngrößen von Schmelzsicherungen} \\\\
            Bemessungsstrom sowie Bemessungsspannung. Der Bemessungsstrom beschreibt jenen Strom, für den die aktiven Teile der Sicherungen gebaut ist. Die Sicherung 
            kann diesem Strom dauernd führen, ohne abzuschalten. Mit der Bemessungsspannung bezeichnet man jenen Strom, für den alle Teile der Sicherung gebaut sind.
            Eine weitere wichtige Größe ist das Bemessungsschaltvermögen. Dies ist jener Kurzschlussstrom, den die Sicherung ohne Gefahr abschalten kann.
            Außerdem gibt die Betriebsklasse über die Verwendbarkeit und über die Strom-Zeit-Kennlinie auskunft.

    \item   \question{Über welche Prüfströme wird die Auslösecharakteristik und Fertigungstoleranz einer Schmelzsicherung festgelegt? Wie sind diese Prüfströme definiert?} \\\\
            Über den großen und kleinen Prüfstrom. Der kleine Prüfstrom ist jener Strom, bei dem die Schmelzsicherung innerhalb der konventionellen Prüfdauer nicht ausschalten darf. Der große Prüfstrom ist jener Strom, 
            bei den die Schmelzsicherung innerhalb der konventionellen Prüfdauer ausschalten muss.

    \item   \question{Erklären Sie Aufbau und Funktion einer Stöpselsicherung (+Skizze)} \\\\
            Die Stöpselsicherung besitzt einen Porzellankörper, auf dem sich an den Enden ein Fuß- und Kopfkontakt befindet.
            Sie verfügt auch über einen Anzeiger. Der Wesentliche Teil ist der \textbf{Schmelzleiter}. Fließt ein zu hoher Strom, so schmilzt dieser Schmelzleiter.
            Der Schmelzleiter ist von Quarzsand umgeben. Der Sicherungseinsatz wird in einen Sockel hineingeschraubt. Diese verfügt über einen Passeinsatz, der das Einschrauben 
            zu hoher Sicherungen verhindert.

    \item   \question{Erklären Sie Aufbau und Funktion einer NH-Sicherung (+Skizze)} \\\\
            NH-Sicherungen bestehen grundsätzlich aus einem Porzellankörper und zwei Kontakten. In der NH-Sicherung befindet sich ein Schmelzleiter umgeben von Sand.
            NH-Sicherungen dürfen nur durch Elektrofachkräfte ersetzt werden. Es muss bei Wechsel ggf. Schutzkleidung getragen werden (Störlichtbogen).

    \item   \question{Worüber gibt die Betriebsklasse einer Schmelzsicherung auskunft?} \\\\
            Die Betriebsklasse gibt über den Verwendbarkeit und über die Zeit-Strom-Kennlinie auskunft.

    \item   \question{Nennen Sie 3 Vorteile eines NH-Trenners gegenüber Stöpselsicherungen.} \\\\
            NH-Schmelzsicherungen können wesentlich höhere Kurzschlussströme abschalten. Die Kontake sind auch hochwertiger und Zuverlässiger:


    \item   \question{Unter welchen Bedingungen werden Schmelzsicherungen gegenüber Leitungsschutzschaltern bevorzugt? (+Beispiele)} \\\\
            Selektivität. Leitungsschutzschalter dürfen nur \textit{zuletzt} angebracht werden. Werden mehrere Leitungsschutzschalter in Serie verwendet,
            so würden bei Kurzschluss alle ausgelöst werden.

    \item   \question{Erklären Sie Aufbau und Funktion eines Leitungsschutzschalters. Wie erfolgt die Auslösung bei einem Leitungsschutzschalter.}\\\\
            Ein Leitungsschutzschalter besitzt einen thermischen und einen magnetischen Auslöser. Der magnetische dient als Kurzschlussschutz, der thermische als 
            Überlastschutz.

    \item   \question{Nennen Sie Vorteile von Leitungsschutzschaltern gegenüber Schmelzsicherungen.}\\\\
            Sie können nach Behebung des Fehlers wieder eingeschaltet werden, was zu weniger Abfall führt. Sie können außerdem nicht geflickt werden,
            was die Sicherheit erhöht. Sie lösen bei Kurzschlüssen eventuell auch schneller aus.

    \item   \question{Über welche Prüfströme  wird die Auslösecharakteristik und Fertigungstoleranz eines Leitungsschutzschalters festgelegt. Wie sind diese Prüfströme definiert?} \\\\
            Es gibt den Nichtauslösestrom, den Auslösestrom und den Sofortauslösestrom. Der Nichtauslösestrom ist jener Strom, bei dem der Leitungsschutzschalter innerhalb der festgelegten Prüfdauer
            nicht aulöst. (1,13IN) Der Auslösestrom ist jener Strom, bei dem der Leitungsschutzschalter innerhalb einer festgelegten Prüfdauer abschaltet. (1,45IN) Bei dem Sofortauslösestrom muss 
            der Leitungsschutzschalter innerhalb 100ms reagieren.

    \item   \question{Wie unterscheiden sich Leitungsschutzschalter der Typen B, C und D in ihrer Auslösecharaktieristik (+Skizze)?} \\\\
            Sie unterscheiden sich nur durch die magnetische Auslösung. Desto höher der Buchstabe, desto höher der Sofortauslösestrom.

    \item   \question{Ein Winkelschleifer hat einen Nennstrom von 6A und nimmt beim Einschalten kurzzeitig den 8fachen Strom auf. Wird ein klagloser Betrieb möglich sein, wenn der Stromkreis mit einem 13A LS-Schalter vom Typ B abgesichert ist oder schlagen Sie einen anderen Leitungsschutzschalter vor? (+Begründung)} \\\\
            Dieser Leitungsschutzschalter wird einen klaglosen Betrieb ermöglichen.
    \item   \question{Worauf muss man bei der Verwendung von LS-Schaltern der Type D im genullten Netz achten?}
    \item   \question{Was versteht man unter der Energiebegrenzungsklasse eines Leitungsschutzschalters? Welche Auswirkung hat eine hohe Energiebegrenzungsklasse?}
\end{enumerate}

\subsubsection{Ausschaltstrom / Selektivität}
Lösen Sie folgende Aufgabenstellungen zum Thema Ausschaltstrom / Selektivität

\begin{enumerate}
    \item   \question{Wie ist der Ausschaltstrom von Überstromschutzorganen definiert?}\\\\
            Der Ausschaltstrom bezeichnet jenen Strom, bei dem der Leitungsschutzschalter innerhalb 100ms zuverlässig abschaltet.
    \item   \question{Was versteht man unter Selektivität von Überstromschutzorgangen? Unter welcher Voraussetzung ist die Selektivität gegeben? Nennen Sie 2 einfache Regeln zur Selektivität.}\\\\
\end{enumerate}